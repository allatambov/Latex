
\documentclass[12pt]{article}
\usepackage[utf8]{inputenc}

% для кириллицы: добавить russian в babel
% добавить кодировку шрифта T2A

\usepackage[T2A]{fontenc}
\usepackage[english, russian]{babel}

% для вставки кода
\usepackage{fancyvrb}

% для одновременного кода и текста
\usepackage{showexpl}
\lstloadlanguages{[LaTeX]Tex} 
\lstset{
    basicstyle=\ttfamily\small,  
    explpreset={numbers=none},
    breaklines=true} 

% для ссылок
\usepackage{hyperref}

 % для цветного текста
\usepackage{color}

% для отступа в первом абзаце
\usepackage{indentfirst} 
\parindent=1.2cm % длина отступа в абзацах

% для продвинутых списков
\usepackage{enumitem} 

% для вставки картинок
\usepackage{graphicx}

% для математики
\usepackage{amssymb}
\usepackage{amsmath}

\begin{document}

\section{Работа с текстом.}

\subsection{Символы в \LaTeX{}.} 

\subsubsection{Специальные символы.} В \LaTeX{} есть специальные символы, которые не получится включить в документ просто так, потому что они будут распознаваться им как служебные знаки, а не как часть текста: 

\begin{center}
\begin{BVerbatim}
% & $ _ # { } ~ \
\end{BVerbatim}
\end{center}

 Чтобы включить их в текст, нужно поставить перед ними обратный слэш, а для самого обратного слэша использовать специальную команду: 
 
 \begin{LTXexample}
 \%, \&, \$, \_, \#, \{, \}, \~{},
  \textbackslash{}.
  \end{LTXexample}

\subsubsection{Тире и кавычки.}

Чтобы получить тире, нужно поставить две черточки рядом, без пробела. 

Дефис: -

Тире: --

Длинное тире: ---

В \LaTeX{} используются два вида кавычек: "лапки" (приняты в англоязычных текстах), и <<ёлочки>> (приняты в русскоязычных текстах). 

 \begin{LTXexample}
`` '' 
<<>> 
  \end{LTXexample}

\subsection{Выделение текста.}

Текст можно выделять по-разному. Во-первых, можно использовать \textit{курсив}. 

\begin{center}
\begin{BVerbatim}
\textit{курсив}
\end{BVerbatim}
\end{center}


Во-вторых, \textbf{полужирное начертание}. 

\begin{center}
\begin{BVerbatim}
\textbf{полужирное начертание}
\end{BVerbatim}
\end{center}

В-третьих, текст можно просто \underline{подчеркивать}. 

\begin{center}
\begin{BVerbatim}
\underline{подчеркивать}
\end{BVerbatim}
\end{center}

Поиграем с начертанием текста. Сделаем \textbf{\textit{полужирный курсив}}:

\begin{center}
\begin{BVerbatim}
\textbf{\textit{полужирный курсив}}
\end{BVerbatim}
\end{center}

Посмотрим на \textsc{научный} шрифт. 

\begin{center}
\begin{BVerbatim}
\textsc{научный} шрифт
\end{BVerbatim}
\end{center}

Добавим \texttt{простой текст}, будто набранный на печатной машинке. 

\begin{center}
\begin{BVerbatim}
\texttt{простой текст}
\end{BVerbatim}
\end{center}

Для того, чтобы включить в документ цветной текст, нужно в преамбуле подгрузить пакет \texttt{color}:

\begin{center}
\begin{BVerbatim}
\usepackage{color}
\end{BVerbatim}
\end{center}

При выделении текста цветом важно не потерять двойные фигурные скобки. Например, так выглядит {\color{red} красный цвет}. 

\begin{center}
\begin{BVerbatim}
Например, так выглядит {\color{red} красный цвет}.
\end{BVerbatim}
\end{center}

Если не поставить внешние фигурные скобки, то весь текст (до конца документа) после \texttt{\textbackslash color\{\}} будет выделен тем  цветом, который указан в \texttt{\textbackslash color\{\}}.

Как можно увидеть в этом файле, в документ можно включать куски кода (любого, \LaTeX{} в том числе), причем код, заключенный в специальное окружение \texttt{BVerbatim} распознается как обычный текст, а не как служебные символы и команды. Еще есть специальное окружение \texttt{LTXexample}, которое позволяет включать код \LaTeX{} и сразу показывать результат его исполнения. О загрузке необходимых пакетов и настройке параметров -- см. преамбулу. А примеров использования этих окружений в данном файле много. Внимание: \texttt{LTXexample} не работает с кириллицей!

\subsection{Положение текста.}

\subsubsection{Отступы и переход на новую строку.} По умолчанию в \LaTeX{} отстутствует отступ в первом абзаце (и в первых абзацах каждого нового раздела). Чтобы добавить его, нужно в преамбуле подгрузить пакет \texttt{indentfirst}. 

\begin{center}
\begin{BVerbatim}
\usepackage{indentfirst}
\end{BVerbatim}
\end{center}

Величину отступа (не только в первом абзаце, но и по всему тексту) можно регулировать. Например, в этом документе длина отступа составляет 1.2 сантиметра, потому что в преамбуле написана соответствующая строчка:

\begin{center}
\begin{BVerbatim}
\parindent{1.2cm}
\end{BVerbatim}
\end{center}

Как можно заметить, \LaTeX{} не реагирует на количество пробелов и переход на новую строку с помощью \textit{Enter}. Чтобы перейти к следующему абзацу, нужно использовать команду \texttt{\textbackslash par}. Пример:

\textit{Первый абзац. Много-много текста. Нужно набрать текста на две-три строчки. Абзац в одну строчку -- не абзац.  Кажется, получилось.}  \par
\textit{А теперь второй абзац. Можно и покороче.}

Как можно заметить, до этого в документе команда \texttt{\textbackslash par} нигде не использовалась, но текст все равно был организован в абзацы с отступами. На самом деле, для перехода на новую строку с сохранением отступа (<<красной строки>>) достаточно в tex-файле оставить пустую строку:

\begin{LTXexample}
line 1

line 2
\end{LTXexample}


Чтобы просто перейти на новую строку, не создавая отступа для нового абзаца, можно использовать двойной обратный слэш. \\ 

\begin{LTXexample}
line 1 \\
line 2
\end{LTXexample}

Увеличивать вертикальные отступы тоже можно. Так выглядят строки с обычным отступом.

\begin{LTXexample}
Sentence 1.  \\
Sentence 2. 
\end{LTXexample}

Например, так будут выглядеть строки со средним отступом (команда \texttt{\textbackslash medskip}): 

\begin{LTXexample}
Sentence 1. \medskip \\
Sentence 2. 
\end{LTXexample}

А так с большим (команда \texttt{\textbackslash bigskip}): 

\begin{LTXexample}
Sentence 1. \bigskip \\
Sentence 2. 
\end{LTXexample}

\subsection{Переход на новую страницу.}

Молча переходим с помощью команды \texttt{\textbackslash newpage}. 

\newpage

\subsubsection{Выравнивание текста.}

Текст можно расположить по центру, выравнять по левому или правому краю. 

\begin{center}
    \textbf{Гениальное эссе на избитую тему}
\end{center}

\begin{flushleft}
Оценка:
\end{flushleft}

\begin{flushright}
Выполнил: \\
студент группы №000 \\
Шаблон Шаблонович Шаблонов  \bigskip \\
\end{flushright}

\textbf{Код для примера выше:}  \bigskip\\

\begin{BVerbatim}
\begin{center}
    \textbf{Гениальное эссе на избитую тему}
\end{center}

\begin{flushleft}
Оценка:
\end{flushleft}

\begin{flushright}
Выполнил: \\
студент группы №000 \\
Шаблон Шаблонович Шаблонов  \bigskip \\
\end{flushright}
\end{BVerbatim}

 \ \ \medskip\\

Более подробно посмотрим на выравнивание, когда будем обсуждать, как создать титульный лист.

\section{Списки.}

Списки бывают нумерованные и ненумерованные. Любой список заключается с специальное окружение, внутри которого пункты списка начинаются с команды \texttt{\textbackslash item}. Окружение для нумерованных списков \texttt{enumerate}, а для ненумерованных -- \texttt{itemize}. \medskip\\

\textbf{Нумерованные списки:}
\begin{LTXexample}
\begin{enumerate}
\item Firstly, 
\item Secondly,
\item Thirdly,
\end{enumerate}
\end{LTXexample}

\textbf{Ненумерованные списки:}
\begin{LTXexample}
\begin{itemize}
\item No number 1
\item No number 2
\item No number 3
\end{itemize}
\end{LTXexample}

По умолчанию в качестве маркера для пунктов используется точка, но при желании его можно поменять -- указать в квадратных скобках сразу после \texttt{\textbackslash item}:
\begin{LTXexample}
\begin{itemize}
\item[-] No number 1
\item[-] No number 2
\item[-] No number 3
\end{itemize}
\end{LTXexample}

Списки можно сочетать -- получать многоуровневые списки:
\begin{LTXexample}
\begin{enumerate}
\item Point 1
\begin{enumerate}
    \item Subsection 1
    \item Subsection 2
    \item 
    \begin{enumerate}
         \item Stop this recursion!
         \begin{enumerate}
             \item Immediately!
         \end{enumerate}
    \end{enumerate}
\end{enumerate}
\item Point 2
\item Pint 3
\end{enumerate}
\end{LTXexample}


Для более продвинутой работы со списками нужен пакет \texttt{enumitem}. Например, можно убрать отступы перед пунктами, добавив опцию \texttt{leftmargin}:
\begin{LTXexample}
\begin{itemize}[leftmargin = *]
\item Point 1
\item Point 2
\item Point 3
\end{itemize}
\end{LTXexample}

\section{Таблицы.}

Для создания таблицы в \LaTeX{} используются два окружения. Окружение \texttt{table} -- для самой таблицы (в нем указывается положение таблицы, ее название и код для таблицы), окружение \texttt{tabular} -- собственно для строк и столбцов.  \medskip


\textbf{Пример простой таблицы:}

\begin{LTXexample}
\begin{table}[ht!] 
\begin{tabular}{lll} 
name & age & height \\
\hline
Ann & 24 & 163 \\
Peter & 10 & 142 \\
\hline
\end{tabular}
\end{table}
\end{LTXexample}

\textbf{Небольшой разбор кода.} В квадратных скобках после \texttt{table} указывается положение таблицы. У \LaTeX{} есть свои представления о том, где располагать таблицу, и эти представления могут не совпадать с нашими желаниями. Поэтому мы можем написать \texttt{[ht!]}, чтобы настойчиво сообщить \LaTeX{}, что таблица должна быть здесь (\texttt{h} -- от \textit{here}). В фигурных скобках после \texttt{tabular} перечисляются буквы \texttt{l}. Число букв равно числу столбцов в таблице, а сами буквы отвечают за выравнивание текста в этих столбцах (\texttt{l} -- по левому краю, \texttt{r} -- по правому краю, \texttt{c} -- по центру). В нашем случае в таблице нет разделителей между столбцами, поэтому буквы просто перечислены подряд. Таблица заполняется по строкам, столбцы в каждой строке отделяются с помощью \&. Для перехода на новую строку таблицы так же используется двойной обратный слэш.  Важно контролировать число знаков \&: их всегда должно быть на один меньше, чем число столбцов в таблице.

\textbf{Еще пример простой таблицы.} Другая таблица, но уже с заголовком (\texttt{\textbackslash caption}) и выравниванием по центру (\texttt{\textbackslash centering}): 

\begin{LTXexample}
\begin{table}[ht!]
\caption{My table.}
\centering
\begin{tabular}{|r|r|} 
\hline
name & age \\
\hline
Ann &  \\
Peter & 10 \\
\hline
\end{tabular}
\end{table}
\end{LTXexample}

Ширину столбцов можно менять -- задавать вручную -- см. после \texttt{tabular}:

\begin{LTXexample}
\begin{table}[ht!]
\caption{This table is also mine.}
\centering
\begin{tabular}{|p{2cm}|p{2cm}|} 
\hline
name & age \\
\hline
Ann &  \\
\hline
Peter & 10 \\
\hline
\end{tabular}
\end{table}
\end{LTXexample}

Для создания больших или более сложных таблиц можно воспользоваться полезным ресурсом для автоматической генерации таблиц онлайн: \href{https://www.tablesgenerator.com/}{https://www.tablesgenerator.com/}. В нем можно создавать таблицы и оформлять их <<привычным>> образом (почти как в Excel или Google Tables): выбирать границы ячеек, объединять ячейки, раскрашивать их и так далее. Сервис будет автоматически преобразовывать готовую таблицу в код \LaTeX{}, который потом можно скопировать в документ (tex-файл).

\section{Картинки.}

После таблиц нам ничего не страшно. Для того, чтобы \LaTeX{} свободно загружал картинки из файлов с разными расширениями (jpg, jpeg, png и другие), в преамбуле нужно догрузить пакет \texttt{graphicx}.

\begin{center}
\begin{BVerbatim}
\usepackage{graphicx}
\end{BVerbatim}
\end{center}

\newpage

Картинки из внешних файлов в \LaTeX{} включаются в текст аналогично таблицам, внутри специального окружения, окружения \texttt{figure}: 

\begin{LTXexample}
\begin{figure}[ht!]
    \centering
    \includegraphics[scale=0.15]{desktop.png}
    \caption{Beautiful picture}
\end{figure}
\end{LTXexample}

\textbf{Пояснения к коду.} Картинка добавляется с помощью команды  \texttt{\textbackslash includegraphics}, в которой в фигурных скобках вписывается название файла. Файл должен лежать рядом с tex-файлом (или быть загружен на Sharelatex). Если \LaTeX{} установлен на компьютер, то файл может лежать в любом месте, не рядом с tex-файлом -- тогда в фигурных скобках нужно будет прописать полный путь к нему (например, \texttt{C:/Users/AllaT/Documents/desktop.png}).  Опция \texttt{scale} в квадратных скобках добавлена, чтобы уменьшить размер картинки (0.15 означает, какую часть от исходного размера картинки составляет новый размер картинки). Можно было бы указать желаемый размер картинки явно, задав ширину и высоту, но тут важно хорошо подобрать значения, чтобы не деформировать картинку (\texttt{scale} изменяет размер пропорионально. а все \texttt{width} и \texttt{hight} -- уже на нашей совести):

\begin{LTXexample}
\begin{figure}[ht!]
    \centering
    \includegraphics[width=2cm, height=4cm]{desktop.png}
    \caption{Beautiful, but deformed picture}
\end{figure}
\end{LTXexample}


И да, картинки (и таблицы тоже) можно добавлять и без окружения \texttt{figure} (для таблиц \texttt{table}). Тогда она будет без названия и выравнивания:

\begin{LTXexample}
 \includegraphics[scale=0.15]{desktop.png}
 \end{LTXexample}


\section{Математика в \LaTeX.}

Вообще формулы в текст можно добавлять и не догружая дополнительных пакетов, но удобнее сразу догрузить пакеты для математических символов и операций (см. преамбулу):

\begin{BVerbatim}
\usepackage{amssymb}
\usepackage{amsmath}
 \end{BVerbatim}

\subsection{Формулы.}

Формулы заключаются в знаки доллара. Если нужно, чтобы формула была посередине, то нужны двойные знаки доллара. Сравним:

\begin{LTXexample}
$P(A) = 2/3$ \\ 
$$P(A) = 2/3$$
 \end{LTXexample}

\subsection{Символы и операторы.} 

\textbf{Умножение:}

\begin{LTXexample}
$a \cdot b$ \\ 
$c \times d$
 \end{LTXexample}
 
 \textbf{Дроби:}

\begin{LTXexample}
 $\frac{2}{3}$ \\
 $\dfrac{\frac{a}{b} \cdot \frac{b}{a}}{8}$
 \end{LTXexample}
 
\textbf{Кванторы:} 

\begin{LTXexample}
 $\forall i $ \\
 $\exists i$
 \end{LTXexample}
 
\textbf{Бесконечность:}

\begin{LTXexample}
$\infty$
 \end{LTXexample}

\textbf{Множества:}

\begin{LTXexample}
$1 \in \{1,2,3\}$ \\ 
$a \in A$ \\
$b \notin A$ \\
$C \subset D$ \\
$C \subseteq D$ \\
$C \ne D$ \\
$A \cap B$ \\
$A \cup B$ \\
$\varnothing$ \\
$\emptyset$
 \end{LTXexample}


\subsection{Греческие буквы.}

\begin{LTXexample}
$\alpha$, $\beta$, \\
$\epsilon$ \\
$\varepsilon$  \\
$\Gamma$, $\Delta$, $\Theta$.
 \end{LTXexample}

\subsection{Степени и индексы.}

\begin{LTXexample}
$y_i = \beta_0 + \beta_1 \cdot x + \beta_2 \cdot x^2$
 \end{LTXexample}

Но если индексы или степени включают больше одной цифры, то нужно не забыть фигурные скобки:

\begin{LTXexample}
$y_{i1} = \beta_{01}+ \beta_{11} \cdot x + \beta_{21} \cdot x^2$
 \end{LTXexample}
 
В качестве индексов можно использовать и буквы:

\begin{LTXexample}
$income_{first}$
 \end{LTXexample}
 
 С пробелом:
 
 \begin{LTXexample}
$income_{first \mbox{ } year}$
 \end{LTXexample}
 
Текст на кириллице тоже можно добавлять в формулы:

\begin{BVerbatim}
$income_\textup{год}$  
\end{BVerbatim}

$income_\textup{год}$ \\

\subsection{Надстрочные символы.}

\begin{LTXexample}
$\bar{x}$ \\
$\hat{y}$ \\
$\tilde{v}$ \\
$\hat{\tilde{y}}$ 
\end{LTXexample}
% 

\subsection{Суммы, интегралы, пределы, производные.}

\textbf{Сумма:}

\begin{LTXexample}
$\sum_{i=1}^{n} x_i$ \\
$\sum\limits_{i=1}^{n} x_i$ 
\end{LTXexample}

\textbf{Произведение:}

\begin{LTXexample}
$\prod_{i=1}^{n} x_i$ \\
$\prod\limits_{i=1}^{n} x_i$ 
\end{LTXexample}

\textbf{Интеграл:}

\begin{LTXexample}
$\int (x + x^2)dx$  \\
$\int\limits_{a}^{b} (x + x^2)dx$ 
\end{LTXexample}

\textbf{Предел:}

\begin{LTXexample}
$$\lim_{x \rightarrow \infty} e^{-x} = 0$$ 
\end{LTXexample}

\textbf{Производная:}

Обычная:

\begin{LTXexample}
$(x+x^2)'$ \\
$\frac{df}{dx}$
\end{LTXexample}

Частная:

\begin{LTXexample}
$\frac{\partial f}{\partial x}$ \\
$\frac{\partial^2 f}{\partial x^2}$
\end{LTXexample}

\subsection{Прочее.}

\begin{LTXexample}
$\mathbb{R}$,  $\mathbb{Z}$, $\mathbb{Q}$ \\
$A \rightarrow B$ \\
$A \leftarrow B$ \\
$A \Rightarrow B$ \\
$A \Leftarrow B$ \\
$A \Leftrightarrow B$
\end{LTXexample}

Про более сложные конструкции (матрицы, системы уравнений), см., например, \href{https://en.wikibooks.org/wiki/LaTeX/Mathematics}{здесь}. 

\href{http://tug.ctan.org/info/symbols/comprehensive/symbols-a4.pdf}{Список} всех символов (не только математических) в \LaTeX{}. 

\end{document}
