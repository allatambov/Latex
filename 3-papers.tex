\documentclass[14pt]{extarticle} % 14 шрифт
\usepackage[utf8]{inputenc}

% для кириллицы
\usepackage[T2A]{fontenc}
\usepackage[english, russian]{babel}

% Times New Roman 
\usepackage{tempora}

% настройка ширины полей
\usepackage{geometry}
\geometry{top=20mm} % верхнее
\geometry{bottom=25mm} % нижнее
\geometry{left=15mm} % левое
\geometry{right=15mm}% правое

% для отступа в первом абзаце
% ширина отступа (красной строки) во всех абзацах
\usepackage{indentfirst}   	
\parindent=1.25cm

% выравнивание по ширине
\sloppy

% межстрочный интервал
\linespread{1.5}

% для ссылок
\usepackage{hyperref}
\hypersetup{
	colorlinks = true,
	linkcolor = black,
  	urlcolor = blue}
   	
% шрифт в названиях разделов - чтобы был полужирный (bfseries),
% но такого же размера, как основной текст в документе (normalsize) 
\usepackage{titlesec}
\titleformat*{\section}{\normalsize\bfseries}
\titleformat*{\subsection}{\normalsize\bfseries}
\titleformat*{\subsubsection}{\normalsize\bfseries}

% добавляем слово Глава для разделов (section) 
% и точки после нумерации подразделов (точка после thesubsection)

\titleformat{\section}[block]{\bfseries}{Глава \thesection.}{1em}{}
\titleformat{\subsection}[block]{\bfseries}{\thesubsection.}{1em}{}
\titleformat{\subsubsection}[block]{\bfseries}{\thesubsubsection.}{1em}{}

% аналогичные действия, но для отображения названий в содержании
\usepackage{titletoc}
\titlecontents{section}
[0pc]
{}
{\bfseries\normalsize Глава \thecontentslabel. } % для разделов section (с номером)
{\bfseries\normalsize} % для разделов secton* (без номера)
{} % номер страницы
[]


\begin{document}

% окружение для титульного листа
% создает отдельную страницу без номера внизу, нумерация пойдет со следующей страницы 
\begin{titlepage} 
\begin{center}
ПРАВИТЕЛЬСТВО РОССИЙСКОЙ ФЕДЕРАЦИИ \\
ФЕДЕРАЛЬНОЕ ГОСУДАРСТВЕННОЕ АВТОНОМНОЕ ОБРАЗОВАТЕЛЬНОЕ \\
УЧРЕЖДЕНИЕ ВЫСШЕГО ОБРАЗОВАНИЯ \\
<<НАЦИОНАЛЬНЫЙ ИССЛЕДОВАТЕЛЬСКИЙ УНИВЕРСИТЕТ \\
<<ВЫСШАЯ ШКОЛА ЭКОНОМИКИ>> \medskip\\
Факультет социальных наук \\
Департамент политической науки \\
\vspace{2cm} % vertical space - вертикальный отступ в 2 см

Фамилия Имя Отчество \\
\textbf{КАКАЯ-ТО СЕРЬЕЗНАЯ ТЕМА НА МНОГО СТРОК ПРО ПОЛИТИЧЕСКУЮ НЕСТАБИЛЬНОСТЬ} \bigskip\\
 Выпускная квалификационная работа -- бакалаврская работа \\
 по направлению подготовки 41.03.04. Политология \\
 студента группы №141 (образовательная программа <<Политология>>)
\end{center}

\begin{flushleft} Рецензент: \end{flushleft} % по левому краю
\begin{flushright} % по правому краю
Научный руководитель: \\
степень, \\
должность \\
Фамилия И.О.
\end{flushright} 
\vfill % vfill - пустота до конца страницы (до последней строки на ней), verticall fill

\begin{center} Москва -- 2017 \end{center}
\end{titlepage}
\newpage

\tableofcontents % содержание

\newpage

\section*{Введение} % * - чтобы не было номера
\addcontentsline{toc}{section}{Введение \hfill \thepage} % добавить строку в содержание
% \hfill \thepage - добавить в содержание номер страницы, чтобы он был не рядом с названием, а в конце страницы
В этом файле в качестве примера собраны отрывки одной старой курсовой работы (еще второго курса, так что все давно и неправда). Цель этого файла: показать, каким образом можно настраивать нумерацию и названия разделов, создавать титульный лист. Так что, все внимание -- на преамбулу и комментарии к коду внутри tex-файла.

Этот файл учитывает многие требования по оформлению курсовых и дипломных работ, но это не означает, что учтены все требования (нужно всегда сверяться с правилами). То же касается титульного листа -- иногда проще взять готовый титульный лист из приложения к требованием по оформлению работ, сконвертировать в pdf и склеить два pdf-файла. 

\section{Проблема определения политической нестабильности и ее влияния на экономический рост}
\subsection{Политическая нестабильность как неустойчивость режима и склонность к изменениям в исполнительной власти}
Одной из основных и наиболее известных работ, посвященных взаимосвязи политической нестабильности и экономического роста является работа «Политическая нестабильность и экономический рост» А.Алесина, С.Озлер, Н.Рубини, Ф.Свагела. Авторы рассматривают политическую нестабильность как склонность к изменениям в исполнительной власти как с помощью законных (конституционных), так и незаконных (неконституционных) способов. Такая политическая нестабильность отрицательно влияет на экономический рост. Это объясняется тем, она увеличивает политическую неопределенность, которая, в свою очередь, оказывает отрицательное влияние на принятие продуктивных экономических решений, объем инвестиций и накопление сбережений. С другой стороны, авторы не исключают возможность влияния самого экономического роста на нестабильность: чем медленнее экономический рост, тем выше уровень нестабильности. Более того, возможен «порочный круг»: медленный рост порождает нестабильность, нестабильность впоследствии ведет к замедлению роста. 
\subsection{Политическая нестабильность как социально-политическая нестабильность в крайних проявлениях}
Одной из известных работ является исследование Р.Барро <<Экономический рост, кросс-страновой анализ>>. Несмотря на то, что автор не концентрирует внимание именно на политической нестабильности и ее концептуализации, он включает ее в свою модель, операционализируя как число революций и coup d’état в год и как число политических убийств на миллион человек в год. Р.Барро обосновывает выбор этих переменных тем,  что при предварительном анализе было выявлено их статистически значимое влияние на экономический рост. При этом политическая нестабильность отрицательно влияет на экономический рост, так как она порождает угрозу правам собственности и ведет к сокращению частных инвестиций, что негативно сказывается на экономическом росте. 
\subsection{Политическая нестабильность как многомерное явление}
Исследование Р.Джон-А-Пина <<Об измерении политической нестабильности и ее влияния на экономический рост>> -- работа, в которой наиболее подробно рассматриваются различные аспекты политической нестабильности. Автор выделяет четыре измерения данного явления: политически мотивированное насилие; массовые протесты; нестабильность <<в рамках режима>> (‘within regime’); нестабильность самого режима (‘of regime’). 
\section{Концептуализация и операционализация основных понятий: политическая нестабильность, экономический рост и развивающиеся страны.}
\subsection{Политическая нестабильность}
Как уже указывалось во введении, в своем исследовании я рассматриваю политическую нестабильность как социально-политическую нестабильность. Исходя из этого, политическая нестабильность  – состояние общего недовольства социальным, политическим и экономическим устройством государства. Политическая нестабильность проявляется в демонстрациях, забастовках и бунтах. Такое понимание нестабильности близко к третьему типу нестабильности, описанному Л.Шовет (\textit{social instability}), о котором говорилось ранее. 
\subsection{Экономический рост}
Под экономическим ростом я, прежде всего, понимаю экономический рост, который выражается в ежегодном росте ВВП на душу населения. 

В качестве показателя экономического роста в работе используется показатель <<ежегодный рост ВВП на душу населения>> из базы Всемирного банка (World Development Indicators).
\subsection{Развивающиеся страны}
Для определения развивающихся стран в работе используется классификация стран Организации Объединенных Наций. Развивающиеся страны (“least developed” и “developing”) – страны с низким валовым национальным доходом на душу населения, с низким значением индекса человеческого капитала (HAI) и высоким уровнем экономической уязвимости (согласно Economic Vulnerability Index).
\section{Анализ данных: <<политическая нестабильность -- иностранные вложения – экономический рост>>}
\subsection{Описание переменных}
Как было указано во введении, для проверки гипотезы о влиянии политической нестабильности в работе используется анализ пространственно-временных данных (time series cross section). В качестве зависимой переменной рассматривается экономический рост – рост ВВП на душу населения. 
\subsubsection{Социально-политическая нестабильность}
Социально-политическая нестабильность разбита на несколько проявлений: демонстрации, забастовки, бунты. 
\subsubsection{Контрольные переменные}
Включены группы контрольных переменных для каждого проявления нестабильности. Для проявления политической нестабильности в  виде демонстраций используются следующие переменные: уровень социального неравенства, наличие политической дискриминации, уровень безработицы, число природных катастроф. 
\subsubsection{Переменные интереса}
Включены <<переменные интереса>> -- <<внешние>> факторы, влияющие на экономический рост: объем внешних инвестиций, объем материальной помощи со стороны отделений Организации Объединенных Наций, объем материальной помощи со стороны международных банков. 
\subsection{Предварительный анализ}
Для предварительного анализа используется общая модель, включающая все факторы, влияющие на экономический рост, а также все бинарные переменные -- проявления нестабильности  (binary treatments) и контрольные переменные, относящиеся к ним. 
\subsection{Регрессионные модели}
\subsubsection{Модель <<наличие демонстраций – экономический рост>>}
\paragraph{Влияние демонстраций (одномоментное)} Уравнение регрессии
\paragraph{Влияние демонстраций с учетом фактора времени и без него} Уравнение регрессии
\subsubsection{Модель <<наличие забастовок – экономический рост>>} Уравнение регрессии
\subsubsection{Модель <<наличие бунтов – экономический рост>>} 
\paragraph{Влияние бунтов (одномоментное)} Уравнение регрессии
\paragraph{Влияние бунтов с учетом фактора времени и без него} Уравнение регрессии 
\newpage
\section*{Заключение}
В данной работе была рассмотрена проблема влияния разных типов политической нестабильности на экономический рост. 
\newpage
\section*{Список используемой литературы и источников}
\newpage
\addcontentsline{toc}{section}{Список используемой литературы и источников \hfill \thepage} \newpage
\addcontentsline{toc}{section}{Заключение \hfill \thepage}
\section*{Приложение А}
\addcontentsline{toc}{section}{Приложение А \hfill \thepage}
\section*{Приложение Б}
\addcontentsline{toc}{section}{Приложение Б \hfill \thepage}

\end{document}