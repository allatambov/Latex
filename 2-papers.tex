\documentclass[12pt]{article}
\usepackage[utf8]{inputenc}

% для кириллицы
\usepackage[T2A]{fontenc}
\usepackage[english, russian]{babel}

% для вставки кода
\usepackage{fancyvrb}

% для ссылок
\usepackage{hyperref}
%\hypersetup{
%	colorlinks = true,
%	linkcolor = black,
%  	urlcolor = blue}
   	
% для отступа в первом абзаце
\usepackage{indentfirst}   	
% ширина отступа (красной строки) во всех абзацах
\parindent=1.25cm

% настройка ширины полей
\usepackage{geometry}
\geometry{top=20mm} % верхнее
\geometry{bottom=25mm} % нижнее
\geometry{left=15mm} % левое
\geometry{right=15mm}% правое

% межстрочный интервал
% \linespread{1.5}

% Times New Roman

%\usepackage{tempora}

% настройка заголовка в таблице 
\usepackage{caption}
%\captionsetup[table]{name=Табл., labelsep=period} % изменим на Табл. и заменим : после номера на .

% для графиков и масштабирования таблиц
\usepackage{graphicx}

% для поворота таблиц
\usepackage[graphicx]{realboxes}

\begin{document}

\begin{center}
{\Large{\textbf{Оформляем эссе, курсовую или дипломную работу}}}
\end{center}

\section{Настройка полей, выравнивания, межстрочных интервалов и отступов}

\subsection{Настройка полей}

Для настройки полей документа понадобится пакет \texttt{geometry}. Для подключения этого пакета напишем соответствующую строчку в преамбуле:

\begin{center}
\begin{BVerbatim}
\usepackage{geometry}
\end{BVerbatim} 
\end{center}

Так как ширина полей обычно одинакова для всего документа, она также определяется в преамбуле. Величину полей можно указывать в разных единицах измерения, укажем в милиметрах:

\begin{center}
\begin{BVerbatim}
\geometry{top=20mm} 
\geometry{bottom=25mm}
\geometry{left=20mm} 
\geometry{right=20mm}
\end{BVerbatim} 
\end{center}

Соответственно, в этом документе ширина верхнего поля 2 сантиметра, нижнего -- 2.5 сантиметра, левого  и правого -- 2 сантиметра.

С помощью пакета \texttt{geometry} можно изменять оформление документа более значительно, например, настраивать размер бумаги, положение основного текста, заголовков и сносок, но это требует более серьезного знакомства с \LaTeX{} и умение работать с документацией (\href{http://mirror.macomnet.net/pub/CTAN/macros/latex/contrib/geometry/geometry.pdf}{ссылка} на документацию по пакету \texttt{geometry}).

\subsection{Выравнивание}

Для того, чтобы выравнять текст по ширине, пригодится команда \texttt{sloppy}, которую, как и все параметры, определяемые для целого документа, указывают в преамбуле:

\begin{center}
\begin{BVerbatim}
\sloppy
\end{BVerbatim} 
\end{center}

Эта команда позволяет растянуть текст по ширине и предотвратить случаи, когда текст выходит за границы полей (да, такое иногда бывает).

\subsection{Межстрочный интервал}

Межстрочный интервал (интерлиньяж) также выставляется в преамбуле. Укажем стандартный для многих работ межстрочный интервал 1.5: 

\begin{center}
\begin{BVerbatim}
\linespread{1.5}
\end{BVerbatim} 
\end{center}

В этом документе эта строчка в преамбуле закомментирована (указана со знаком \%), поэтому сохранен межстрочный интервал по умолчанию -- одинарный.

\subsection{Отступы}

По умолчанию \LaTeX{} опускает отступ в первом абзаце документа (а еще в первом абзаце раздела/подраздела и так далее). Это можно исправить, подгрузив пакет с логичным названием \texttt{indentfirst}. 

\begin{center}
\begin{BVerbatim}
\usepackage{indentfirst}
\end{BVerbatim} 
\end{center}

Кроме того, можно выставить ширину отступа -- ширину <<красной строки>> в каждом абзаце:

\begin{center}
\begin{BVerbatim}
\parindent=1.25cm
\end{BVerbatim} 
\end{center}

\section{Настройка шрифтов}

Многие работы требуется выполнять, используя шрифт Times New Roman, 14 кегль. По умолчанию \LaTeX{} не использует шрифт Times New Roman и часто с его использованием для текста на кириллице возникают проблемы. Для удобства догрузим пакет \texttt{tempora}, который позволяет использовать Times New Roman для греческого и кириллицы. Чтобы увидеть этот документ, выполненным Times New Roman, нужно раскомментировать соответствующую строчку в преамбуле:

\begin{center}
\begin{BVerbatim}
\usepackage{tempora}
\end{BVerbatim} 
\end{center}


В \LaTeX{} есть еще пакет \texttt{times}, но он работает корректно только для текстов на латинице. Если в тексте есть кириллица, то \texttt{times} не только не изменит шрифт, но и уберет заодно весь курсив, полужирное начертание и прочие попытки выделить текст.

По умолчанию в \LaTeX{} используется 10 шрифт (10pt). В данном документе размер шрифта равен 12 пунктам, как следует из первой (самой главной) строчки в преамбуле:

\begin{center}
\begin{BVerbatim}
\documentclass[12pt]{article}
\end{BVerbatim} 
\end{center}

Кажется, для того, чтобы изменить размер шрифта, достаточно изменить число в квадратных скобках. На самом деле, это не совсем так: такой способ будет работать только для шрифтов размера в 10, 11 и 12 пунктов, так как именно эти размеры по умолчанию поддерживает тип документа \texttt{article}. Для того, чтобы использовать другие размеры, типичные для нас и нетипичные для \LaTeX{}, нужно иначе указать тип документа:

\begin{center}
\begin{BVerbatim}
\documentclass[14pt]{extarticle}
\end{BVerbatim} 
\end{center}

Можно изменить таким образом первую строчку в преамбуле этого документа и посмотреть, как увеличится шрифт.

\section{Оформление ссылок и сносок}

\subsection{Ссылки}

По умолчанию \LaTeX{} не воспринимает написанные гиперссылки как ссылки, он считает их за обычный текст. Со всеми вытекающими последствиями: \LaTeX{} может ругаться на символы, которые часто используются в гиперссылках (\&, \textbackslash, \_). Чтобы использовать гиперссылки в документе, нужно догрузить пакет \texttt{hyperref}:

\begin{center}
\begin{BVerbatim}
\usepackage{hyperref}
\end{BVerbatim} 
\end{center}

После загрузки этого пакета ссылки в документе будут кликабельными. Иногда бывает, что по ссылке кликнуть можно, но она не открывается. Тогда нужно просто кликнуть на нее правой клавишей и выбрать <<Копировать адрес ссылки>>. Как обычно, когда мы включаем ссылку в документ, мы указываем саму гиперссылку и текст для нее. Например, так:

\begin{center}
\begin{BVerbatim}
\href{https://ru.sharelatex.com/learn}{Документация от ShareLaTeX}
\end{BVerbatim} 
\end{center}

Эта строка дает нам ссылку на \href{https://ru.sharelatex.com/learn}{документацию от ShareLaTeX}. По умолчанию гиперссылки не выделяются цветом, вокруг них добавляется рамка. Но это, конечно, можно изменить, настроив нужные параметры в преамбуле:

\begin{center}
\begin{BVerbatim}
\hypersetup{
	colorlinks = true,
	linkcolor = black,
   	urlcolor = blue}
\end{BVerbatim} 
\end{center}   	
   	
Написанный код делает следующее: 

\begin{itemize}

\item разрешает выделение ссылок цветом: \texttt{colorlinks = true}
\item выбирает черный цвет для обычных ссылок, то есть ссылок внутри документа (на разделы, таблицы, рисунки): \texttt{linkcolor = black}
\item выбирает синий цвет для гиперссылок: \texttt{urlcolor = blue}

\end{itemize}
   	
В данном документе эти строки в преамбуле закомментированы, чтобы увидеть изменения (синие ссылки без рамочек), их нужно раскомментировать. О других опциях при работе с ссылками, см., например, \href{https://ru.sharelatex.com/learn/Hyperlinks}{здесь}.
   	
\subsection{Сноски}

Чтобы добавить сноску в документ, используется команда \texttt{\textbackslash{footnote}\{\}}, где в фигурных скобках указывается текст сноски. По умолчанию сноски имеют сквозную нумерацию\footnote{ Номер сноскам присваивается друг за другом по всему документу.} и нумеруются арабскими цифрами. Про стили оформления сносок для всего документа и отдельных сносок -- см. \href{https://ru.sharelatex.com/learn/Footnotes}{здесь} и \href{https://en.wikibooks.org/wiki/LaTeX/Footnotes_and_Margin_Notes}{здесь}.

Размер шрифта в сносках всегда меньше, чем в самом документе, но при желании его тоже можно настраивать. 

\section{Цитирование}

В \LaTeX{} есть специальное окружение для больших цитат \texttt{quote}. Текст, который находится внутри этого окружения, не отличается от остального текста, просто выделяется отступами (как по вертикали, так и по горизонтали). Процитируем Википедию:

\begin{quote}
TeX -- система компьютерной вёрстки, разработанная американским профессором информатики Дональдом Кнутом в целях создания компьютерной типографии. В неё входят средства для секционирования документов, для работы с перекрёстными ссылками. Многие считают TeX лучшим способом для набора сложных математических формул. В частности, благодаря этим возможностям, TeX популярен в академических кругах, особенно среди математиков и физиков.

Название произносится как <<тех>> (от греч. $\tau\acute{\varepsilon}\chi\nu\eta$ -- <<искусство>>, <<мастерство>>). В написании буква E опущена ниже T и X. В самой программе название форматируется как \TeX{}.
\end{quote}

Для того, чтобы оформлять цитирование более концептуально (с указанием источника с ссылкой на список литературы), нужно более внимательно ознакомиться с оформлением библиографии в \LaTeX{}. Почитать про нее можно \href{https://en.wikibooks.org/wiki/LaTeX/Bibliography_Management}{здесь} и \href{https://ru.sharelatex.com/learn/Bibliography_management_in_LaTeX}{здесь}. Наверное, если число источников не очень большое (и если у нас нет единой базы источников, которые мы цитируем то в одной работе, то в другой), можно обойтись и без специального оформления библиографии, и оформить список литературы как обычный нумерованный/ненумерованный  список.

\section{Оформление рисунков и таблиц}

\subsection{Таблицы}

Согласно требованиям ко многим письменным работам, таблицы в документе должны быть пронумерованы и иметь название. Обычно используется сквозная нумерация: каждая таблица имеет номер, соответствующий ее порядковому номеру в документе. Другими словами, если у нас есть четыре таблицы, две в первом разделе, две во втором, они будут иметь номера 1, 2, 3, 4, а не 1.1 и 1.2, 2.1 и 2.2. 

В \LaTeX{} к заголовку таблицы (\texttt{caption}) автоматически добавляется ее порядковый номер в документе:

\begin{table}[ht!]
\centering
\caption{Индексы стран}
\begin{tabular}{|c|c|}
\hline
Страна & id \\
\hline
Австрия & 1 \\
Бельгия & 2 \\
Великобритания & 3 \\
Германия & 4 \\
\hline
\end{tabular}
\end{table}

Если текст написан на русском языке (догружен русский язык в \texttt{babel} и подходящая кодировка), название таблицы начнается со слова <<Таблица>>. Часто вместо полного названия требуется использовать сокращенное, например, <<Табл.>>. Чтобы это сделать, нужно в преамбуле подгрузить пакет \texttt{caption} и указать новое наименование таблицы:

\begin{center}
\begin{BVerbatim}
\usepackage{caption}
\captionsetup[table]{name=Табл.} 
\end{BVerbatim} 
\end{center}

Таким же образом можно изменить двоеточие после номера таблицы (перед названием) на точку:

\begin{center}
\begin{BVerbatim}
\captionsetup[table]{name=Табл., labelsep=period} 
\end{BVerbatim} 
\end{center}

Другие возможные разделители номера  названия, а также прочие опции, см. в \href{http://mirrors.mi.ras.ru/CTAN/macros/latex/contrib/caption/caption-rus.pdf}{документации} (кстати, на русском языке). Чтобы посмотреть на изменения в этом документе, нужно раскомментировать (убрать \%) указанную выше строку в преамбуле.

Для того, чтобы название таблицы указывалось внизу таблицы, нужно просто строку с \texttt{caption} указать после окружения \texttt{tabular}. Код для таблицы \ref{table:id2}: \medskip\\


\begin{BVerbatim}
\begin{table}[ht!]
\centering
\begin{tabular}{|c|c|}
\hline
Страна & id \\
\hline
Дания & 5 \\
Люксембург & 6 \\
Монако & 7 \\
Нидерланды & 8 \\
\hline
\end{tabular}
\caption{Индексы стран}
\label{table:id2}
\end{table}
\end{BVerbatim} 

\begin{table}[ht!]
\centering
\begin{tabular}{|c|c|}
\hline
Страна & id \\
\hline
Дания & 5 \\
Люксембург & 6 \\
Монако & 7 \\
Нидерланды & 8 \\
\hline
\end{tabular}
\caption{Индексы стран}\label{table:id2}
\end{table}

Часто приходится ссылаться на таблицы в тексте. Чтобы делать это автоматически, для таблицы нужно создать метку (\texttt{label}), с помощью которой мы будем потом обращаться к таблице. На метку для таблицы 2 можно посмотреть в примере выше. Обращение к таблице осуществляется с помощью команды \texttt{ref\{\}}, в фигурных скобках которой указывается метка, присвоенная таблице:

\begin{center}
\begin{BVerbatim}
\ref{table:id2}
\end{BVerbatim}
\end{center}

Может возникнуть вопрос: зачем так все усложнять? Ведь можно просто в тексте написать <<таблица 2>> и никаких меток не понадобится. Можно. Но тогда в случае, если в текст будет добавлена еще одна таблица перед этой, номер данной таблицы изменится -- это уже будет не таблица 2, а таблица 3. И придется искать все упоминания таблицы 2 в тексте и вручную исправлять. Если ссылки на таблицы создавались с помощью \texttt{label} и \texttt{ref}, никаких лишних действий не потребуется, нумерация в ссылках изменится автоматически.

Метка, созданная для таблицы, может использоваться и для ссылки на страницу, на которой эта таблица находится. Например, можно написать следующее: 
\begin{center}
\textit{В таблице \ref{table:id2} на стр. \pageref{table:id2} перечислены индексы (id) стран, которыми они закодированы в базе данных, используемой в исследовании.}
\end{center}

Иногда таблицы оказываются слишком большими и не помещаются на страницу. В случае, если таблица не многостраничная (<<чуть-чуть не влезает на страницу>>), можно ее ужать, отмасштабировать с помощью \texttt{\textbackslash scalebox} из пакета \texttt{graphicx} (пакет догружается в преамбуле). Перед \texttt{tabular} нужно указать, насколько сильно мы сжимаем объект. Например, хотим получить таблицу размера 80\% от исходной  -- исходная слишком широкая: \medskip\\

\begin{BVerbatim}
\scalebox{0.8}{
\begin{tabular}{|c|c|c|c|}
\hline
Voice and Accountability & Political Stability and Absence of Violence & 
	Control of Corruption & Rule of Law\\ \hline
1.8                      & 1.9                                         & 1.4                  & 1.6 \\ \hline
1.5                      & 1.6                                         & 1.9                   & 2.0\\ \hline
-2.1                     & -2.3                                        & -1.8                  &-2.4\\ \hline
\end{tabular}}
\end{BVerbatim}

Указываем множитель 0.8 (для 80\% от исходного размера) и не теряем фигурные скобки (особенно закрывающую после \texttt{\textbackslash end\{tabular\}}). А вот и сама таблица:

\begin{table}[ht!]
\centering
\caption{WGI random values}
\scalebox{0.8}{
\begin{tabular}{|c|c|c|c|}
\hline
Voice and Accountability & Political Stability and Absence of Violence & Control of Corruption & Rule of Law\\ \hline
1.8                      & 1.9                                         & 1.4                  & 1.6 \\ \hline
1.5                      & 1.6                                         & 1.9                   & 2.0\\ \hline
-2.1                     & -2.3                                        & -1.8                  &-2.4\\ \hline
\end{tabular}}
\end{table}

Еще один способ уместить таблицу на одной странице -- повернуть ее. Это особенно актуально для широких регрессионных таблиц. Чтобы повернуть таблицу нам потребуется пакет \texttt{realboxes}.

\begin{center}
\begin{BVerbatim}
\usepackage[graphicx]{realboxes}
\end{BVerbatim}
\end{center}

 Повернем предыдущую таблицу на 90 градусов -- вместо \texttt{\textbackslash scalebox} напишем \texttt{\textbackslash Rotatebox\{90\}} (разместим ее отдельно на следующей странице):\medskip\\

 
\begin{BVerbatim}
\Rotatebox{90}{
\begin{tabular}{|c|c|c|c|}
\hline
Voice and Accountability & Political Stability and Absence of Violence 
	& Control of Corruption & Rule of Law\\ \hline
1.8                      & 1.9                                         & 1.4                  & 1.6 \\ \hline
1.5                      & 1.6                                         & 1.9                   & 2.0\\ \hline
-2.1                     & -2.3                                        & -1.8                  &-2.4\\ \hline
\end{tabular}}
\end{BVerbatim}

 \newpage 
 
\begin{table}[ht!]
\centering
\caption{WGI random values}
\Rotatebox{90}{
\begin{tabular}{|c|c|c|c|}
\hline
Voice and Accountability & Political Stability and Absence of Violence & Control of Corruption & Rule of Law\\ \hline
1.8                      & 1.9                                         & 1.4                  & 1.6 \\ \hline
1.5                      & 1.6                                         & 1.9                   & 2.0\\ \hline
-2.1                     & -2.3                                        & -1.8                  &-2.4\\ \hline
\end{tabular}}
\end{table}

\newpage

Про мностраничные таблицы можно почитать \href{http://texdoc.net/texmf-dist/doc/latex/tools/longtable.pdf}{здесь} и \href{https://texblog.org/2011/05/15/multi-page-tables-using-longtable/}{здесь}. 

\subsection{Рисунки}

Нумерация рисунков и присваивание им заголовков/меток выглядит точно так же, как и в случае с таблицами. Но, в отличие, от таблиц, в заголовке рисунка по умолчанию фигурирует уже сокращенное название <<Рис.>>. Пример: \medskip\\

\begin{center}
\begin{BVerbatim}
\begin{figure}[ht!]
\centering
\caption{Рабочий стол}
\label{im:galaxy}
\includegraphics[scale=0.4]{desktop.png}
\end{figure}
\end{BVerbatim}
\end{center}

\begin{figure}[ht!]
\centering
\caption{Рабочий стол}
\label{im:galaxy}
\includegraphics[scale=0.4]{desktop.png}
\end{figure}

Ссылаться на рисунки можно так же, как и на таблицы. Например: \textit{На рис.\ref{im:galaxy} что-то изображено}. Да, небольшая деталь про метки и ссылки: если название метки в \texttt{\textbackslash ref\{\}} не совпадает с меткой в \texttt{\textbackslash label\{\}} (опечатались в названии, и такой метки не существует), то ничего страшного не произойдет -- на месте номера таблицы/рисунка будут красоваться знаки вопроса.
 
\section{Структура документа: разделы}

Уже на примере этого документа можно посмотреть, каким образом в \LaTeX{} создаются разделы документа. Есть четыре основных уровня разделов:

\begin{center}
\begin{BVerbatim}
\section{} % раздел
\subsection{} % подраздел
\subsubsection{} % подподраздел
\paragraph{} % параграф
\end{BVerbatim}
\end{center}

Названия разделов и подразделов можно включить в автоматически собираемое содержание. Параграфы в содержание не включаются. Добавим содержание документа с помощью команды \texttt{\textbackslash tableofcontents}:

\tableofcontents

Если строчка для настройки ссылок в преамбуле (с \texttt{colorlinks=true}) осталась закомментированной, то все пункты содержания будут в рамочках, так как они тоже являются ссылками внутри документа.

Все разделы в \LaTeX{} (\texttt{section}, \texttt{subsection} и \texttt{subsubsection}) автоматически нумеруются. Но можно добавить и раздел без номера, поставив звездочку перед названием раздела в фигурных скобках:

\begin{center}
\begin{BVerbatim}
\section*{Раздел без номера} 
\end{BVerbatim}
\end{center}

\section*{Раздел без номера}
\addcontentsline{toc}{section}{Раздел без номера}

Номер перед последним разделом не добавился, как мы и хотели. Но есть проблема: в содержании этот раздел тоже не появится, при сборке содержания все разделы <<со звездочкой>> игнорируются. Чтобы добавить раздел без номера в содержание, нужно после названия этого раздела дописать еще одну строчку:

\begin{center}
\begin{BVerbatim}
\section*{Раздел без номера} 
\addcontentsline{toc}{section}{Раздел без номера}
\end{BVerbatim}
\end{center}

Для чего вообще могут понадобиться ненумерованные разделы? Например, для введения и заключения. Все главы пронумерованы (Глава 1, Глава 2, Глава 3), а вводная или заключительная части нет. Но для тогда и разделы должны нумероваться соответствующим образом! Включать слово <<Глава>> и убирать номер перед самой главой. Об этом -- см. третий tex-файл.
 
\end{document}